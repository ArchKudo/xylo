% thesis.tex
%
%
% Sample document to demonstrate the uwthesis document class, for theses
% in the University of Wales, Aberystwyth.
%
% You should also have received the files uwthesis.cls, mybib.bib and
% README together with this file, if not complain to the person you got
% it from.
%
% Richard Huss <rah94@aber.ac.uk>, 17 Dec 1997.
% Sorry, the README seems to have disappeared.
% Edel Sherratt, <eds@aber.ac.uk> 15 Jan 2018, 23 July 2018

\documentclass{main}
\usepackage{graphicx}
\usepackage{subfigure}
\usepackage{ebgaramond}
% \usepackage[none]{hyphenat}

% Stolen from tax se

% Drop later

\usepackage{tokcycle}[2021-03-10]
\usepackage{xcolor}
\newcounter{wordcount}
\newcounter{lettercount}
\newcounter{wordlimit}
\newif\ifinword
% USER PARAMETERS
\newif\ifrunningcount
\newif\ifsummarycount
\def\limitcolor{red}
\setcounter{wordlimit}{0}
%%
\makeatletter
% \tc@defx is like \def, but expands the replacement text once prior to assignment
\newcommand\addtomacro[2]{\tc@defx#1{#1#2}}
\newcommand\changecolor[1]{\tctestifx{.#1}{}{\addcytoks{\color{#1}{}}%
  \tc@defx\currentcolor{#1}}}
\makeatother
\newcommand\dumpword{%
  \addcytoks[1]{\accumword}%
  \ifinword\stepcounter{wordcount}
    \ifrunningcount\addcytoks[x]{$^{\thewordcount,\thelettercount}$}\fi
    \ifnum\thewordcount=\value{wordlimit}\relax\changecolor{\limitcolor}\fi
  \fi%
  \inwordfalse
  \def\accumword{}}
\newcommand\addletter[1]{%
  \tctestifcatnx A#1{\stepcounter{lettercount}\inwordtrue}{\dumpword}%
  \addtomacro\accumword{#1}}
\xtokcycleenvironment\countem
  {\addletter{##1}}
  {\dumpword\groupedcytoks{\processtoks{##1}\dumpword\expandafter}\expandafter
    \changecolor\expandafter{\currentcolor}}
  {\dumpword\addcytoks{##1}}
  {\dumpword\addcytoks{##1}}
  {\stripgroupingtrue\def\accumword{}\def\currentcolor{.}
    \setcounter{wordcount}{0}\setcounter{lettercount}{0}}
  {\dumpword\ifsummarycount\tcafterenv{%
    \par(Wordcount=\thewordcount, Lettercount=\thelettercount)}\fi}

% Drop later

\usepackage{authordate1-4} % Use this to get Harvard/author-date style referencing

% Include the nolof and/or nolot option for no lists of figures and
% tables respectively, if required. For example,
%  \documentclass[nolof,nolot]{uwthesis}
% will produce a thesis with neither. Useful if you have no figures
% or tables.

% Load any additional LaTeX packages you need here

% Information about this thesis

\title{Comparing and contrasting the protein embedding landscape with commercially important enzymes haplotypes recovered from meta-genomes.}    % Your thesis title
\author{Girish Vyas}                      % Your name
\dept{Department of Computer Science} % Your department
\supervisor{Dr. Wayne Aubrey}                   % Your supervisor's name
\module{CSM9060}                      % Your module code (CSM9060, CHM9360, etc.)
\degree{MSc. Data Science}             % Your degree (MSc Computer Science (Software Engineering), MSc Data Science, MSc Statistics for Computational Biology)

\begin{document}
\setlength{\parindent}{0pt}
% Set the default line spacing. If you want double spacing the
% default, use \doublespacing instead of \onehalfspacing.

\onehalfspacing

% But the preface stuff is single-spaced
\begin{singlespace}

  % Generates the title page, signature pages etc.
  \beforepreface

  % 'Preface' type sections are introduced with \prefacesection.
  \prefacesection{Abstract}

  % Replace with your abstract
  % The abstract stands alone as a very short version of the dissertation.

  % The abstract should state the
  % - [x] scope and
  % - [x] principal objectives of the project, describe the
  % - [x] methods,
  % - [x] summarize the results and state the principal conclusions.

  % (Max. 300 words.)

  % - [x] Xylitol
  % - [x] Xylose reductase
  % - [x] Biotech production
  % - [x] Metagenomes
  % - [x] Hungate
  % - [x] Recovering haplotypes using Hansel\&Gretel
  % - [ ] Using protT5 embedding
  % - [x] Source code

  % TODO Grammarly

\countem

  \textbf{Motivation}: Xylitol is a~~n alditol which can be used as~~ food substitute to sugar but with only 40\% the calories. It is currently mass produced by conversion of purified xylose obtained from raw biomass materials by the process of catalytic hydrogenation using Raney Nickel. This process consumes high energy, producing chemical byproducts and hence an eco-friendly industrial fermentation method is preferred. However there is a need to find and engineer enzymes like the xylose reductase gene from Candida Tropicalis to increase yield and make it economically viable. This research proposes analyzing the complex microbial fermentation community present in the rumen of ruminants like cattle using metagenomics.\\
  \textbf{Scope}: In this research we investigate variation in Xylose Reductase (XYL1) gene, how they align to the Hungate cattle rumen metagenome project, and compare their protein embedding space.\\
  \textbf{Results}: 240 Illumina HiSeq 4000 (TODO verify) were aligned to 30 XYL1 genes using bowtie2 alignment tool to recover 1047 reads. This along with their variants were passed to the python package Gretel for recovering the complete set of genetic variants called the ``haplotype" although in this instance the results were unfavourable and lack of variants made it unable to recover the complete haplotype.\\
  \textbf{Availability}: The reference workflow run on Slurm HPC, the aligned reads, variants, the output of running the Gretel algorithm for attempting to recovering the haplotypes (TODO prott5) is available as an MPL-2.0 licensed Git repository at https://github.com/ArchKudo/xylo

\endcountem
(\thewordcount{} words and \thelettercount{} letters)

  \prefacesection{Declaration of originality}

  I confirm that:\\
  \begin{itemize}
    \item This submission is my own work, except where clearly indicated.
    \item I understand that there are severe penalties for Unacceptable Academic Practice, which can lead to loss of marks or even the withholding of a degree.
    \item I have read the regulations on Unacceptable Academic Practice from the University’s Academic Quality and Records Office (AQRO) and the relevant    sections of the current Student Handbook of the Department of Computer Science.
    \item In submitting this work, I understand and agree to abide by the University’s regulations governing these issues.
  \end{itemize}
  \vspace{10pt}
  \textbf{Name:} Girish Vyas\\
  \textbf{Date:} 2023/09/29
  \vspace{50pt}

  \begin{center}
    \textbf{Consent to share this work}
  \end{center}
  \begin{itemize}
    \item By including my name below, I hereby agree to this thesis being made available to other students and academic staff of the Department of Computer Science, Aberystwyth University.
  \end{itemize}

  \vspace{10pt}
  \textbf{Name:} Girish Vyas\\
  \textbf{Date:} 2023/09/29



  \prefacesection{Acknowledgment}

  % Replace with your acknowledgments

  %\prefacesection{\ }  % A blank section title, just to force a new page
  % Replace with your dedication

  \null\vskip1.5in
  \begin{center}

    % TODO
    To whoever has the patience to read this :-)

    This section is customary, but not obligatory.  It makes a brief statement of thanks to those who have helped.

  \end{center}

  % Insert the table of contents and lists of tables and figures
  \afterpreface

  % Double (or one-half) spacing from here onwards
\end{singlespace}

% Your thesis goes here!
\chapter{Introduction}

\countem
% ~3000 words


% - [ ] Background to the project,
%     - [x] Fermentation
%     - [ ] Enzyme
%     - [ ] Directed Evolution
%     - [ ] xylitol production
        % - [x] chemical production
        % - [x] biological production
%     - [ ] metagenomes
%         - [ ] chromosome, allele, haplotypes, ploidy
%         - [ ] haplotype algorithms
%         - [ ] hungate
%     - [ ]

\section{Background}

% Rewritten

% [1] Y. Xu, P. Chi, M. Bilal, and H. Cheng, “Biosynthetic strategies to produce xylitol: an economical venture,” Appl Microbiol Biotechnol, vol. 103, no. 13, pp. 5143–5160, Jul. 2019, doi: 10.1007/s00253-019-09881-1.

% [2] D. Dodd, R. I. Mackie, and I. K. O. Cann, “Xylan degradation, a metabolic property shared by rumen and human colonic Bacteroidetes,” Molecular Microbiology, vol. 79, no. 2, pp. 292–304, 2011, doi: 10.1111/j.1365-2958.2010.07473.x.

% [3] D. Umai, R. Kayalvizhi, V. Kumar, and S. Jacob, “Xylitol: Bioproduction and Applications-A Review,” Frontiers in Sustainability, vol. 3, 2022, Accessed: Jul. 25, 2023. [Online]. Available: https://www.frontiersin.org/articles/10.3389/frsus.2022.826190

% What is Xylitol and advantages
% Hard words parenteral fibrinogen

Xylitol is a naturally occurring sugar-substitute with with various applications in the pharmaceutical and food industry. It is a five carbon sugar alcohol (alditol) with chemical formula $C_5H_{12}O_5$. The measure of sweetness of Xylitol is comparable to sugar yet boasts only 40\% the calories with ~2.4 calories/gram calorific content compared to sucrose/glucose sugars at 4.0 calories/gram. Having been know since the 1890's to organic chemists, it wasn't until 1975 when the company Finnish Sugar Co. Ltd. started its mass production due to dicovery of it insulin independent metabolic regulation. Making it very appealing to patients with diabetes for preventing irregular blood sugar level spikes. While also boasting anti-carcinogenic and anti-inflammatory properties and used for treatment of osteoporosis, ear infections, and dental care for tooth rehardening and remineralizaiton. Other various benefits include its inertness to amino acids making it suitable for intravenous parenteral nutrition and as a stabilizing agent for proteins extracted from cell membranes and fibrinogen pastes in the human plasma. Lastly, it has also seen adoption in the food industry particularly for manufacturing confectionary, due to its endothermic cooling effect (34.8 calories/gram), its flexibility in improving texture, and preserving nutrition of food proteins. Despite having some shortcomings like having a laxative effect on consumption of more than the 15 grams/day prescribed limit by FDA, Department of Energy (DOE) deservingly recognizes Xylitol as "Top value-added" biomass chemical utilizing fermentation of abandoned plants.

% Add xylose v/s sucrose infographic

% Natural limited sources
% Hard words: D-mannitol, D-sorbitol, hemicellulosic, lignocellulose

Similar to other sugar substitutes like D-mannitol, D-sorbitol, Xylitol is produced in small quantities by fruit plants, vegetables, hardwood trees, as well in husks and stalks of plants. Animals including humans also produce xylitol as a part of glucose metabolism, where adult humans may produce upto 15 grams of xylitol each day. However most available sources have limited extraction capability except hemicellulosic vegetation primarily birchwood and corn-cob along with some other flora which is chemically processed to separate the lignocellulose containing xylans (xylose and arabinose).

% Add xylan structure diagram

% Current xylitol production methods
% Hard words: interchain linkage, hydrolysis, monomeric, hydrolyzate, hydrogenation, chromatography
Originating in Finland, hemicellulose is pretreated with dilute acids to break the interchain linkage by the process of acid hydrolysis reduction to get monomeric sugars. The hydrolyzate is purified to get D-Xylose. It is followed by raney nickel catalyzed hydrogenation at a pressure of 8-10 MPascals temperature of $80^{\circ}$ to $140^{\circ}$ celsius to produce Xylitol. Finally, Xylitol is itself purified and crystallized with a conversion yield of 50-60\% from the hydrolyzate. Although, this yield has improved over the years by using pretreatment methods like 1\% sulphuric acid at $120^{\circ}$ celsius to extract 99\% of the xylose additionally, with the use of better catalysts such as using noble metal like Ruthenium, or non-noble like cobalt and silica. However the dependence of chemicals on synthesis is unsustainable due to the high energy requirements, as well as the use of expensive and potentially toxic metals makes the process ecologically unsafe and requires complex chromatography and recrystrallization to remove impurities prone to contamination and safety issues. Thus there is a growing interest for a high yield and environmentally friendly biological fermentation alternative for Xylitol production.

% Fermentation
% Hard words: Pichia stipitis, xylose reductase, gene encoding, acetyl groups, phosphorylated, nadph, enzyme
% Add pentose phospate pathway diagram

Biotechnological conversion of xylose to xylitol was first reported using Saccharomyces cerevisiae (yeast) with modified Pichia stipitis' (CBS 6054) xylose reductase (XR, EC 1.1.1.307) gene encoding. Not only did it have higher yield than regular yeast strains, bioconversion requires a simple fermentation process with reduced temperature and pressure requirements and lack metal debris impurities facilitating easier purification. In general the bioconversion of xylose happens in the presence of enzyme xylose reductase which converts xylose to xylitol. Hemicellulose fraction itself doesn't need to be hydrolyzed using chemical methods and can undergo autohydrolysis at a temperature around $200^{\circ}$ celsius where the acetyl groups get hydrolyzed and break the bonds. However, conversion of xylose using modified micro-organisms is governed by the pentose phosphate metabolic pathway. Yeast and fungi in this pathway convert xylose to xylitol however, if it isn't secreted from the cell it is converted into xylulose due to oxidation by xylitol dehydrogenase which gets further phosphorylated by the presence of xylokinase. To prevent this continous supply of NADPH/NADH is required as opposed to NADP/NAD. The conversion may also be intervened by the presence of xylose isomerase which directly converts xylose to unwanted xylulose. Recent developments allow improving this, by using only the enzyme xylose reductase (XR) instead of whole cells of micro-organisms and can thus theoretically achieve 100\% yield.









% - [ ] motivation
%     - [ ]
% - [ ] leading to project aims and objectives.
% - [ ] What problem was tackled?
% - [ ] Why was that problem tackled?

% - [ ] Clear statement of project aim and objectives.
\section{Aim}

% - [ ] How (in outline) was the problem tackled?
%     - [ ] alignment
%     - [ ] variant calling
%     - [ ] gretel
%     - [ ] prott5

\section{Solution}

% - [ ] Guide to subsequent chapters.

\endcountem
(\thewordcount{} words and \thelettercount{} letters)

\chapter{Literature review}

\countem
% ~4000 words

The literature review is all about the related knowledge that you are building on.  Similar products and related research are usual.

Remember to use your own words and to show relevance to your project aim.
The literature review will refer extensively to the bibliography.  Harvard (author-date) and IEEE reference styles are usual in Computer Science, but the only real rule is that you should use a consistent style.

Here is an example reference to inky matters~\cite{Jones2010}.

Refereed articles are generally considered to have the greatest authority, but for a Computer Science project you are also likely to cite other sources, such as technical documents, user manuals, standards documents, web pages and books.

When you cite a web source, make sure to include the date of access.

\endcountem
(\thewordcount{} words and \thelettercount{} letters)

\chapter{Reporting on the project -- the core chapters}

\countem
% ~5000 words

Reporting on the project will normally require more than one chapter.
A development project is likely to have chapters addressing
- [ ] requirements,
- [ ] design,
- [ ] implementation,
- [ ] testing and
- [ ] packaging if a plan-driven method is used.  If an agile approach is taking, you might have a chapter for
- [ ] each sprint or iteration.
- [ ] file formats
- [ ] choice of tools
- [ ] wall time optimization
- [ ] positive control?
- [ ] descriptive analysis
- [ ] Quality control
- [ ] embedding space
Other kinds of project will have chapters that are appropriate to the project in question.

You are likely to include diagrams or images in your core chapters.

% \begin{figure}
%   \label{fig-website}
%   \begin{center}
%     \includegraphics[width=30em]{DynamicWebsite.png}
%     \caption{Structure of a dynamic website (Edel Sherratt)}
%   \end{center}
% \end{figure}

\endcountem
(\thewordcount{} words and \thelettercount{} letters)

\chapter{Critical Evaluation}

\countem
% ~2000 words

- [ ] Alternative datasets
- [ ] Statistics
- [ ] Why no haplotypes recovered
- [ ] Reflection


In this chapter (it won’t be chapter 4, but probably chapter 6 or 7 once all the core chapters have been added.)

The critical evaluation consists of a discussion, leading to conclusion.  It is an essential part of a master’s degree.

It shows that you can not only carry out a substantial piece of work, but that you can reflect on it, and think critically about how you might have done it better.

Examiners view the critical evaluation as very important.

\endcountem
(\thewordcount{} words and \thelettercount{} letters)

\chapter{Conclusion}

\countem
% ~500 words

A brief summary of all that has gone before.

May include some directions for future work.


\endcountem
(\thewordcount{} words and \thelettercount{} letters)

% ~14.5k total

\bibliographystyle{ieee}
\bibliography{main}  % Pathname of your bibTeX file

\end{document}
